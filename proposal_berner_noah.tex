\documentclass[10pt]{article}
\usepackage[utf8]{inputenc}
\usepackage{hyperref}

\usepackage{geometry}
 \geometry{
 a4paper,
 total={170mm,257mm},
 left=20mm,
 top=20mm,
 }
 
 %package foir the bibliography
\usepackage[backend=biber, style=alphabetic, maxcitenames=2, sorting=ynt]{biblatex}
\addbibresource{bibliography/proposal.bib}

\begin{document}
\begingroup
\centering
{\Large Semester Project}\\
\LARGE \textbf{
	Printed Circuit Board Design for\\
	Direct Digital Synthesis
}\\~\\
\Large Student: Noah Berner~\\~\\

\large Matt Grau, Christopher Axline\\
ETH Zurich\\
{
\centering
}
\endgroup

\section*{Introduction}
Lasers are one of the most important tools in trapped-ion experiments.
They are employed in multiple facets of the experiment, from photoionization, laser cooling, state preparation via optical pumping,  state readout via laser-induced fluorescence, as well as driving internal state transitions.
These tasks rely on precise control of the laser amplitude, frequency, and phase, possibly even in real-time during the timescale of a single experimental cycle. % several ms
One device commonly used for this purpose is an acousto-optic modulator (AOM).
By modulating a piezoelectric transducer attached to an optical medium with a radiofrequency drive, the resulting soundwaves in the material create a periodically varying index of refraction.
Laser light passing through the optical medium is then diffracted by this grating, in a phenomenon known as Bragg diffraction.
This can be used to modulate the laser intensity, steer beam pointing, as well as shift the laser frequency and phase.
Direct digital synthesis (DDS) is a technique that is well suited for generating the control signals for the AOM devices, as it allows experiments to easily produce spectrally pure radiofrequency tones with well-defined frequency and phase.

\section*{Existing solutions}
Experiments in the Trapped Ion Quantum Information (TIQI) group at ETH typically will use several dozen AOMs, each driven by DDS.
While many critical beamlines used to drive quantum logic gates require real-time control during a single experimental sequence, many beam paths are operated at essentially fixed frequencies and phases, and only need to be updated slowly, out of the loop of the main sequence.
One of the signal generators used for these "fixed frequency" beamlines is a custom DDS solution comprising a Raspberry Pi, an Analog Devices AD9959 DDS evaluation board, a custom interface PCB with power delivery, and preamplifiers on each output.
While this system is capable of synthesizing the required radiofrequency signal, it is assembled from a multitude of components, creating an opportunity for streamlining.
It is also not yet fully connected to the control system of the experiment.

\section*{Objectives and goals}
The main goal of this semester project is to design a custom PCB that houses all the components needed for DDS, which principally are:
\begin{enumerate}
	\item The Analog Devices AD9959 DDS ASIC and supporting passive components.
	\item A digital interface comprising a CPU or microcontroller which communicates with the DDS chip over SPI, and with the experiment either by USB or Ethernet.
	\item Analog preamplifier stages to increases the output power to the several dBm level.
	\item Power supplies for the DDS digital and analog stages, preamplifiers, and digital interface.
\end{enumerate}

Secondary goals are that the PCB is relatively simple to assemble, with as many of the components being pre-soldered onto the PCB at production.
Finally, time permitting, the DDS board should integrate well into the TIQI group control system. There are two stages of integration. The first is that parameters like amplitude, frequency, and phase

\newpage

\section*{Work Plan}

The project will take place over 7 weeks at the beginning of the ETH Autumn Semester 2020.

\begin{enumerate}
	\item
	      Finalize the specification of the PCB and learn the basics of PCB design.

	\item
	      Study the reference design of the AD9959 evaluation board to reproduce it as the first part of the new PCB design.

	\item
	      Add the connections for the other elements to the PCB (microcontroller, Ethernet or USB connection, power delivery, and preamplifiers).

	\item
	      Get the prototype manufactured and solder on any missing components.

	\item
	      Test the prototype to ensure it works.

	\item
	      Improve the design and repeat the last two steps.

	\item
	      Once the prototype meets the specifications, develop the interface to connect the microcontroller to the control system.

	\item
	      Test the whole DDS system with experiments.

	\item
	      Write up documentation.

\end{enumerate}

\printbibliography[heading=bibintoc,title={Bibliography}]

\end{document}
