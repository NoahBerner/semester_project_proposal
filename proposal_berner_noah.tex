\documentclass[10pt]{article}
\usepackage[utf8]{inputenc}
\usepackage{hyperref}

\usepackage{geometry}
 \geometry{
 a4paper,
 total={170mm,257mm},
 left=20mm,
 top=20mm,
 }
 
 %package foir the bibliography
\usepackage[backend=biber, style=alphabetic, maxcitenames=2, sorting=ynt]{biblatex}
\addbibresource{bibliography/proposal.bib}

\begin{document}

\begingroup  
  \centering
{\Large Semester Project}\\
  \LARGE \textbf{
Printed Circuit Board Design for\\
Direct Digital Synthesis
}\\~\\
  \Large Student: Noah Berner~\\~\\
 
  \large Matthew Grau, Christopher Axline\\
ETH Zurich\\
{
\centering
}
\endgroup

\section*{Introduction}
In trapped-ion experiments, lasers with precise frequency and phase are often used as tools. They are, for example, used to drive the ion's internal state transitions or for cooling the ions below a point so that they can be trapped. For these tasks, it is of paramount importance that the frequency of the laser matches the predetermined value for the experiment to a high degree of precision to avoid unexpected behavior and errors.

One way to achieve a high precision on the frequency of laser light is to use acousto-optic modulators (AOMs) to adjust the original frequency of the laser by a small amount. AOMs use the acousto-optic effect to change the frequency of light using sound waves at radio-frequency. These driving sound waves can be generated by direct digital synthesis (DDS). Thus the TIQI group at ETH needs DDS devices for their trapped ion experiments.

\section*{Existing solutions}
Currently, the TIQI group uses a system of a Raspberry Pi, an AD9959 evaluation board, a custom interface PCB with power delivery, and preamplifiers on each output to synthesize the sound waves. While this system is capable of synthesizing the desired sound waves, it is assembled from different components, which takes time and creates an opportunity for errors to occur. It is also not yet fully connected to the control system of the experiment.

\section*{Objectives and goals} 
The main goal of this semester project is to design a custom PCB that houses all the components needed for DDS on a single board. This reduces the possibility of errors during assembly and operation. It can also be easier to mass-produce such a device since there are companies that specialize in PCB assembly.

Connecting the new DDS-board directly to the control system is also part of the project. A direct connection to the control system can reduce the time to set up experiments if some parameters need to be changed. It can also allow for new experiments because the DDS can be directly controlled by data from the experiment in a feedback loop.


\newpage

\section*{Work Plan}

\begin{enumerate}
	\item Learn the basics of PCB design.

	\item Study the design of the AD9959 to reproduce it as the first part of the new PCB design.

	\item Add the connections for the other elements to the PCB (microcontroller, Ethernet or USB connection, power delivery, and preamplifiers)

	\item Get the prototype manufactured and solder on any missing components.

	\item Test the prototype to ensure it works.

	\item Improve the design and repeat the last two steps.

	\item Once the prototype meets the specifications, develop the interface to connect the microcontroller to the control system.

	\item Test the whole DDS-system with experiments. 

\end{enumerate}

\printbibliography[heading=bibintoc,title={Bibliography}]

\end{document}
